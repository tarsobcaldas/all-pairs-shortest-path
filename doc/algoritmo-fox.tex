\section{Paralelizando a multiplicação através do algoritmo de Fox}

O \emph{algoritmo de Fox} nos permite dividir a tarefa de multiplicação
da matriz em diversos processos fazendo multiplicações de matrizes menores
em cada um. Para isso, dividimos a matriz em um \emph{grid}, isto é,
uma malha com divisões iguais ao número de processos. Por exemplo,
na matriz \(M\), poderiamos ter

\[
	M_0 =
	\begin{bmatrix}
		0 & 2 & 9 \\
		0 & 0 & 0 \\
		9 & 2 & 0
	\end{bmatrix},\quad
	M_1 =
	\begin{bmatrix}
		5  & 6 & 14 \\
		0  & 0 & 0  \\
		14 & 6 & 5
	\end{bmatrix},\quad
	M_2 =
	\begin{bmatrix}
		4 & 6 & 4 \\
		3 & 5 & 3 \\
		4 & 6 & 4
	\end{bmatrix},
	M_3 =
	\begin{bmatrix}
		0 & 1 & 9 \\
		8 & 0 & 8 \\
		9 & 1 & 0
	\end{bmatrix}.
\]

Com \ac{mpi}, podemos definir o grid como um \emph{struct}
\cinline/GRID_TYPE/, como mostra o \cref{lst:grid-type}. O grid que criamos deve possuir
dimensão \(q\timesq\), onde \(q=\sqrt{p}\) e \(p\) é o número de processos. Dizemos
que \(q\) é a \emph{ordem} do \emph{grid}.

\begin{listing}
	\cfile[firstline = 6, lastline = 15]{../mtrxop.h}
	\caption{Definição do tipo \cinline/GRID_TYPE/ no arquivo \file{mtrxop.h}}%
	\label{lst:grid-type}
\end{listing}

Aqui definimos três comunicadores, um para o grid, um para linhas e outro para colunas,
além de definirmos o número de processos, a ordem  do grid, além de ter a informação
da linha e coluna no grid. Usamos então para inicar o \emph{grid} a função \cinline{setupGrid(GRID_TYPE)}, %chktex 36
definida no \cref{lst:setup-grid}.

\begin{listing}
	\cfile[firstline=85,lastline=105]{../mtrxop.c}
	\caption{Inicialização do \emph{grid} definida em \file{mtrxop.c}.}%
	\label{lst:setup-grid}
\end{listing}

Por fim, definimos no \cref{lst:fox} função que aplica o algoritmo de Fox para
efetuar a multiplicação de matrizes. Este algoritmo escolhe uma submatriz \(A\) de
cada linha e a envia para os outros processos daquela linha, e então cada processo
multiplica a matriz recebida com uma matriz \(B\) presente previamente (inicialmente 
igual a \(A\)). Cada processo então envia a matriz \(B\) para o processo acima no \emph{grid}.

\begin{listing}
	\cfile[firstline=108,lastline=127]{../mtrxop.c}
	\caption{Algoritmo de Fox.}%
	\label{lst:fox}
\end{listing}
