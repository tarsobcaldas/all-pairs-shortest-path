\section{Conclusão}

Com este trabalho foi possível chegar à conclusão que o tamanho do problema é fundamental quando queremos
encontrar o ponto ótimo de eficiência da distribuição de processos usando \ac{mpi}. Quando o problema
é pequeno demais, o custo da comunicação não compensa as possíveis vantagens do paralelismo, e mesmo
quando consideramos problemas maiores, nem sempre mais máquinas ou mais processos significa uma melhoria
no tempo de execução. Por este motivo, antes de se decidir uma configuração para executar um programa em
paralelo, é sempre importante ter os \emph{benchmarks} adequados para o tipo e tamanho de problema que estamos
lidando.
