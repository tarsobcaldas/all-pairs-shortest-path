\pagebreak

\section{Execução do programa}

Para executar o programa, basta compilar o projeto usando \verb|make|, e então
executar como seguem as instruções no \cref{lst:help}

\begin{listing}
  \small
	\begin{verbatim}
  Usage: mpirun -np <processes> shortest-path [options]
      -i <file> --input <file>    Set matrix input file
      -o <file> --output <file>   Specify custom output name
      -h --help                   Display this help message
      -t --timing-only            Don't print the solution, only the timing
      -r --random-matrix <size>   Use random matrix as input
                                  (only if no input file provided)
  \end{verbatim}
	\caption{Funcionalidade do programa \file{shortest-path}.}%
	\label{lst:help}
\end{listing}

É necessário que o número de processos seja um quadrado perfeito, a fim de
que estes possam ser organizados em um \emph{grid}.

Podemos rodar o programa tanto utilizando um arquivo com uma matriz em
que a primeira linha contém o número de nós do grafo, ou então gerar uma matriz
aleatória de tamanho arbitrário. Em ambos os casos, as matrizes planificadas
em um \emph{array} \cinline{totalMatrix[]} de tamanho \(n\times n\), que é então populado nas matrizes
\(A\) de cada processo através de \cinline{MPI_Scatter()}. %chktex 36
Compiamos então a matriz \(A\) para \(b\) e \(C\), e podemos
aplicar o algoritmo de Fox até que tenhamos a matriz de distância, 
como ilustra o \cref{lst:fox-loop}.

\begin{listing}
  \cfile[firstline=155, lastline = 165]{../main.c}
  \caption{Loop do algoritmo Fox sobre as matrizes \(A\) e \(B\) de cada processo.}%
  \label{lst:fox-loop}
\end{listing}

Terminamos usamos \cinline{MPI_Gather()} para copiar as matrizes de volta %chktex 36
para o array \cinline{totalMatrix[]} e temos o nosso resultado.
